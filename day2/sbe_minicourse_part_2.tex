\documentclass[aspectratio=169, xcolor=dvipsnames, 11pt]{beamer}
\usetheme{moloch}
\molochset{
  progressbar = frametitle,
  block = fill
}
% Use beamer's page number in head/foot template
\setbeamertemplate{footline}[frame number]
\setbeamersize{text margin left=0.5cm,text margin right=0.5cm}

% from https://tex.stackexchange.com/a/441705/36296
\makeatletter
\addtobeamertemplate{date}{\centering}{}
\addtobeamertemplate{institute}{\centering}{}
\patchcmd{\beamer@@tmpl@title}{\raggedright}{\centering}{}{}
\patchcmd{\beamer@@tmpl@author}{\raggedright}{\centering}{}{}
\setlength{\moloch@titleseparator@linewidth}{2pt}
\setlength{\moloch@progressinheadfoot@linewidth}{1.5pt}
\setbeamercolor{progress bar}{fg=FGVBlue, bg=white}
\makeatother

% ----- Packages -----
\usepackage{graphicx}
\usepackage{amsmath, amssymb}
\usepackage{subcaption}
\usepackage{booktabs}
\usepackage{xcolor} % For colored bullets
\usepackage{appendixnumberbeamer}
\usepackage{emoji}
\setemojifont{Apple Color Emoji}
% TikZ for decorative braces in slides
\usepackage{tikz}
\usetikzlibrary{decorations.pathreplacing,positioning,calc}

% ----- Metadata -----
\title[]{ML in Economics and Finance: Where do We Go Now? - Part I}

\author[]{
  \textbf{Raul Riva}
}
\institute[]{
  \textcolor{FGVBlue}{FGV EPGE}
  \vspace{0.75cm}
}
\date[]{December, 2025 \\ \vspace{1cm} INSPER - São Paulo}

% ----- Custom commands -----
\newcommand{\R}{{\mathbb{R}}}
\renewcommand{\Pr}{{\mathbb{P}}}
\newcommand{\E}{{\mathbb{E}}}
\newcommand{\var}{{\mathrm{var}}}
\newtheorem{prop}[theorem]{Proposition}

% ----- Defining colors -----
\usepackage[dvipsnames]{xcolor}
\definecolor{slateblue}{RGB}{45, 62, 80}
\definecolor{KelloggPurple}{HTML}{4E2A84}
\setbeamercolor{palette primary}{bg=slateblue, fg=white}
\definecolor{FGVBlue}{RGB}{0, 114, 188}
\setbeamercolor{title separator}{bg=FGVBlue, fg=FGVBlue}
\setbeamercolor{background canvas}{bg=white}

% ----- Hyperlinks and Bibliography -----
\usepackage[backend=biber,style=authoryear,maxcitenames=2,mincitenames=1,maxbibnames=99,uniquename=false,uniquelist=false]{biblatex}
\ExecuteBibliographyOptions{maxcitenames=2,mincitenames=1,uniquelist=false}
\addbibresource{../references.bib}
\usepackage{hyperref}
\hypersetup{
  colorlinks=true,
  linkcolor=FGVBlue,
  urlcolor=FGVBlue,
  citecolor=FGVBlue
} % close \hypersetup
\DeclareFieldFormat{labelyear}{#1}
\DeclareFieldFormat{shorthand}{#1}
\DeclareFieldFormat{extraalpha}{#1}
\AtEveryCitekey{\color{FGVBlue}}
\DeclareNameFormat{labelname}{%
    \nameparts{#1}%
    \usebibmacro{name:family}{\namepartfamily}{\namepartgiven}{\namepartprefix}{\namepartsuffix}%
    \usebibmacro{name:andothers}}
\let\cite\textcite
\renewcommand*{\bibfont}{\tiny}

% Make section titles slateblue
\makeatletter
% Preserve starred sections and optional arguments while coloring section titles.
\let\oldsection\section
\renewcommand{\section}{\@ifstar\star@section\@section}
\newcommand{\@section}[2][]{%
    \if\relax\detokenize{#1}\relax
        \oldsection{\textcolor{slateblue}{#2}}%
    \else
        \oldsection[#1]{\textcolor{slateblue}{#2}}%
    \fi
}
\newcommand{\star@section}[1]{\oldsection*{\textcolor{slateblue}{#1}}}
\makeatother


%---------------- Document Start ----------------%
\begin{document}

\begin{frame}{Flight Plan {\normalfont\emoji{airplane}}}
    \begin{enumerate}
        \item What is ML, anyway?\tikz[remember picture,baseline] \node (it1) {};
        \item Causality in High Dimensions\tikz[remember picture,baseline] \node (it2) {};
        \item (Seriously) Heterogeneous Partial Effects\tikz[remember picture,baseline] \node (it3) {};
        \item Solving Large-Scale General Equilibrium Models\tikz[remember picture,baseline] \node (it4) {};
    \end{enumerate}

    % Decorative braces indicating Today / Tomorrow on the right
    \begin{tikzpicture}[remember picture,overlay]
        \coordinate (todayAnchor) at ($(it2.east)+(0.45cm,-0.2cm)$);
        \coordinate (tomorrowAnchor) at ($(it4.east)+(0.45cm,-0.2cm)$);

        \draw[decorate,decoration={brace,amplitude=6pt,mirror},line width=1pt]
            (todayAnchor |- it2.south) -- ($(todayAnchor |- it1.north)+(0,0.3cm)$)
            node[midway,right=12pt]{\textbf{Yesterday}};

        \draw[decorate,decoration={brace,amplitude=6pt,mirror},line width=1pt]
            (tomorrowAnchor |- it4.south) -- ($(tomorrowAnchor |- it3.north)+(0,0.3cm)$)
            node[midway,right=12pt]{\textbf{Today}};
    \end{tikzpicture}
\end{frame}

\section{Heterogeneous Partial Effects}

\begin{frame}{Motivation}
Let $Y$ be an outcome and $X, Z$ be features (covariates). We frequently want to approximate
\begin{equation*}
    h(x, z) \equiv \E[Y | X=x, Z=z]
\end{equation*}
and the \textbf{partial effects}
\begin{equation*}
    \frac{\partial}{\partial x} h(x, z) = \frac{\partial}{\partial x} \E[Y | X=x, Z=z].
\end{equation*}
\begin{itemize}
    \pause
    \item This is a prediction problem after all!
    \pause
    \item Approach 1: impose a parametric model for $h$, e.g. linear regression. Pros and cons?
    \item Approach 2: use fully nonparametric methods. Pros and cons?
    \pause
    \item The third way is the charm: a bit of structure, a bit of ML!
\end{itemize}
\end{frame}

\begin{frame}{Example I - Heterogenous Treatment Effects}
\begin{itemize}
    \item Outcomes $Y_i$ depend on a treatment $X_i \in \R$ and covariates $Z_i \in \R^p$;
    \item The dose $X_i$ depends on observables $Z_i$;
    \item Potential outcomes $Y_i(x)$ for each dose $x \in \R$;
    \pause
    \item The conditional average effect of increasing the dose is $\tau(x, z) \equiv \frac{\partial}{\partial x} \E[Y(x) | Z=z]$;
    \item If $\E[Y(x) | X=x, Z=z] = \E[Y(x) | Z=z]$ (common assumption in the literature), then 
    \[
    \tau(x, z) = \frac{\partial}{\partial x} \E[Y | X=x, Z=z] = \frac{\partial h(x,z)}{\partial x}
    \]
\end{itemize}
\end{frame}

\begin{frame}{Example II - Grouped Heterogeneity}
Consider the following model:
\begin{equation*}
    Y_i = \alpha(Z_i) +  X_i' \beta + \varepsilon_i, \quad \E[\varepsilon_i | X_i, Z_i] = 0,  
\end{equation*}
\begin{itemize}
    \item $X_i$ affects $Y_i$ homogeneously;
    \item Intercept $\alpha(Z_i)$ varies with $Z_i$, maybe in a highly nonlinear way;
    \item Since $Z_i$ and $X_i$ can be correlated, this can affect inference about $\beta$;
    \item \cite{Bonhomme2015} studied how democracy affects national income using this model;
    \item In their case: $\alpha(Z_i)$ is constant across groups but $Z_i$ defines membership;
    \item In our notation: $h(z, x) = \alpha(z) + x' \beta$
\end{itemize}
\end{frame}


\begin{frame}[standout]
    Questions?
\end{frame}

\begin{frame}[standout]
    \begin{center}
       Thank you!\\
        See you tomorrow, stay tuned!
    \end{center}
\end{frame}

% --- Appendix ---
\appendix

\begin{frame}[standout]
  \begin{center}
    \Huge
    Appendix and References
  \end{center}
\end{frame}

\begin{frame}{References}
    \footnotesize
    \printbibliography[heading=none]
\end{frame}

\end{document}
